\chapter{Conclusões e Trabalhos Futuros}
\label{ch:conclusoes}

\section{Conclusões}

Este trabalho desenvolveu no domínio do tempo e no domínio da frequência as formulações necessárias ao cálculo das interferências inter-simbólica e inter-portadora, as quais afetam os símbolos recebidos em sistemas DMT com prefixo cíclico insuficiente. De maneira distinta do que usualmente se encontra na literatura (a exemplo de \cite{Malkin08,perodling2002}), estas formulações foram desenvolvidas levando em consideração um cursor (ponto de referência para inicio da recepção) não nulo, devido à consideração da sincronização simbólica.

Com estas formulações, tornou-se possível eliminar a distorção por prefixo cíclico insuficiente  (ISI e ICI) nos símbolos recebidos, no intuito de diminuir a degradação de desempenho resultante desta distorção. A eliminação ou mitigação da distorção, por sua vez, foi desenvolvida de duas maneiras distintas: a partir de manipulação de símbolos no transmissor, na qual os símbolos a serem transmitidos são transformados (pré-codificados) de maneira que sejam recebidos sem distorção PCI, e a partir da manipulação de símbolos no receptor, na qual a distorção presente nos símbolos recebidos é removida (equalizada).

Especificamente, desenvolveu-se um esquema de mitigação de ISI e ICI a partir de dois métodos: a mitigação de ICI a partir de pré-codificação de símbolos no transmissor em conjunto com a equalização de ISI no receptor, e a mitigação conjunta de ISI e ICI somente a partir de pré-codificação de símbolos no transmissor. Através de ambos, tornou-se possível superar significativamente o máximo desempenho de sistemas DMT com prefixo cíclico. Este fato pôde ser observado a partir dos resultados das simulações de desempenho apresentadas nas Figuras \ref{fig:simulacao_ch6_data_rate_135dbm_noise} e \ref{fig:simulacao_ch11_data_rate_135dbm_noise}, os quais evidenciaram a discrepância entre o desempenho dos sistemas DMT com o esquema de mitigação de distorção PCI proposto em relação aos sistemas DMT ordinários, com diferentes escolhas de prefixos cíclicos (dentre elas a escolha ótima), bem como a aproximação dos sistemas DMT com mitigação de distorção PCI em relação ao limite superior da taxa de bits.

Para alcançar estes resultados, fez-se uma criteriosa análise das alterações de potência induzidas pela operação de pré-codificação, da qual resultaram métodos de controle de potência dos símbolos pré-codificados. No Capítulo~\ref{sec:mitigacao_pci}, no intuito de assegurar que a densidade espectral de potência dos símbolos pré-codificados se mantenha nos níveis dos símbolos originais, fez-se uso da normalização da matriz pré-codificadora através de um fator escalar. Este, no entanto, devido ao seu tratamento simplista ao problema, pode acarretar em significativas perdas de SNR no sistema, as quais resultariam em um despenho indiscutivelmente inferior em relação ao sistemas sem o esquema de mitigação proposto (como pôde-se observar no caso tratado na Figura~\ref{fig:simulacao_ch11_data_rate_135dbm_noise_noNormalization}). Diante disto, desenvolveu-se técnicas mais detalhistas de normalização, com destaque para a técnica denominada normalização por coluna, a qual, efetivamente tornou unitária a norma Euclidiana de todas as linhas da matriz de pré-codificação (conforme simulação apresentada na Figura~\ref{fig:normas_ericsson9_100m_100Mhz}), o que se configura como um pré-requisito para a preservação da densidade espectral de potência do símbolo na operação de pré-codificação.

Finalmente, este trabalho apresentou o desenvolvimento de novas técnicas de redução de complexidade da matriz de pré-codificação. Através da técnica denominada ``pré-codificação de subsímbolos em tons de frequência positiva'' (Seção~\ref{sec:hermitian_precoder}), apresentou-se uma análise matemática que possibilita a exploração da garantida simetria Hermitiana dos símbolos pré-codificados para reduzir o esforço computacional envolvido nas operações de pré-codificação e equalização a $25\%$ do esforço originalmente necessário à esta operações (como mostrado na Figura~\ref{fig:comp_operacoes_esc_reais}). Similarmente, a técnica denominada ``pré-codificação no domínio do tempo'' propiciou a redução do esforço computacional para a operação de pré-codificação a percentuais iguais ou menores que $25\%$ do esforço original (conforme Figura~\ref{fig:comp_operacoes_esc_reais_td_worst}). Particularmente, esta técnica apresentou a valiosa vantagem de poder eliminar amostras insignificantes da resposta impulsiva para reduzir a quantidade de colunas não nulas da matriz de pré-codificação no domínio do tempo e, finalmente, reduzir significativamente a complexidade da pré-codificação. Neste caso, dependendo do tamanho da resposta impulsiva do canal, a quantidade de operações necessárias à pré-codificação pôde ser reduzida a valores apreciavelmente abaixo de $25\%$, como evidenciou a suposição da Figura~\ref{fig:comp_operacoes_esc_reais_timedomain}.

\section{Trabalhos Futuros}

A teoria desenvolvida neste trabalho propicia um promissor potencial para estudos futuros no âmbito da redução de complexidade e controle de potência de algoritmos de pré-codificação. Com as técnicas apresentadas, pôde-se reduzir a complexidade da pré-codificação a niveis tão baixos quanto $8.8\%$ da complexidade original (Figura~\ref{fig:comp_operacoes_esc_reais_timedomain}) e controlar com êxito a potência dos símbolos pré-codificados. No entanto, ainda há um longo caminho para a redução de complexidade tomar niveis tão baixos a ponto da utilização do prefixo cíclico tornar-se indiscutivelmente inválida e para assegurar a robustez dos algoritmos de controle de PSD.

Em especial, a técnica de pré-codificação no domínio do tempo apresenta potencial para redução de complexidade a níveis extremamente reduzidos, a partir da consideração criteriosa a respeito do número de amostras da resposta impulsiva do canal. Além disso, existe uma indicação de que a operação de pré-codificação possa ser simplificada através de algoritmos semelhantes aos algoritmos de decimação no tempo ou decimação na frequência \cite{oppenheim1998}, os quais são utilizados na operação de FFT e IFFT para reduzir a complexidade da operação de maneira extraordinária (como ilustra a Figura \ref{fig:cmacs_fft}). Esta indicação advém da simetria existente na matriz de pré-codificação, a qual é, em parte, semelhante à simetria existente na matriz DFT.

Em relação ao controle de potência, tem-se como objetivo para trabalhos futuros assegurar a funcionalidade incondicional dos algoritmos. O algoritmo de normalização por coluna, a despeito de ter proporcionado o controle efetivo da PSD do símbolo pré-codificado, apresenta a condição de dominância diagonal significativa na matriz de pré-codificação. Diante disto, este algoritmo pode não ser funcional para cenários diferentes de aplicação da técnica, o que motiva a busca por uma solução mais generalista ao problema de escolha dos ganhos de escalamento das colunas da matriz de pré-codificação. Esta solução seria de grande interesse não somente para a pré-codificação de distorção PCI, mas também para as técnicas de mitigação de \emph{crosstalk} entre os usuários de sistemas DMT (denominada \emph{Vectoring} \cite{cendrillon2007}), para a qual a possibilidade de utilizar uma técnica similar à normalização por coluna está sendo estudada \cite{itutq4046}.