
\chapter{Introdução}

Em engenharia de telecomunicações, existe uma busca incessante por utilizar mais eficientemente os recursos finitos (físicos) dos sistemas, como a largura de banda disponível, a potência máxima de transmissão e a capacidade de processamento, com o objetivo final de disponibilizar com confiabilidade a maior taxa de transmissão de dados possível. Seguindo esta tendência, os sistemas multitom discreto (DMT) se consolidaram como a alternativa preferível em canais de comunicação com significativa dispersão, devido à capacidade que o sistema DMT apresenta em explorar a largura de banda do canal de forma mais eficiente, em comparação à um sistema de comunicação de portadora única, e à simplicidade de sua implementação. 

A simplicidade dos sistemas DMT está fortemente relacionada à simplicidade na remoção dos efeitos do canal (equalização) nos símbolos recebidos. Esta, por sua vez, ampara-se no princípio de que os símbolos recebidos equivalem ao resultado de uma convolução circular com a resposta impulsiva do canal, o que permite que suas transformadas discretas de Fourier (DFT) sejam expressas como o produto entre as transformadas discretas de Fourier do símbolo transmitido e do canal, isto é, permite expressar o símbolo como o produto entre um fator devido ao símbolo transmitido e um fator devido ao canal, o qual é fácilmente eliminado pelo receptor. Portanto, nas condições de equivalência entre o símbolo recebido e o resultado de uma convolução circular, é possível utilizar um equalizador extremamente simples para retirar os efeitos do canal e recuperar o símbolo transmitido.

Contudo, a equivalência entre o símbolo recebido e a convolução circular do símbolo transmitido com a resposta impulsiva no canal nem sempre é garantida. O elemento que assegura este comportamento, o qual consiste em um intervalo de amostras adicionais anexado aos símbolos transmitidos (denominado prefixo cíclico), pode não ter o número de amostras necessário para tal propósito. Neste caso, os símbolos recebidos ficam sujeitos a distorções, a saber, a interferência inter-símbolica (ISI) e a interferência inter-portadora (ICI), às quais atribui-se a denominação de distorção por prefixo cíclico insuficiente (distorção PCI).

A distorção PCI pode impactar severamente no desempenho de um sistema DMT, o que motiva a busca imediata de sua solução: tornar o prefixo cíclico suficiente, através do aumento do seu número de amostras. No entanto, o aumento do número de amostras no prefixo cíclico torna o símbolo transmitido mais duradouro, o que implica em uma redução na taxa de transmissão de símbolos. Assim, existe um compromisso entre aumentar o número de amostras do prefixo cíclico, visando torná-lo suficiente, e diminuir a taxa de transmissão de símbolos do sistema, 

Este compromisso está ilustrado na Figura \ref{fig:tradeoff_cp_bitrate}, a qual apresenta um resultado típico do percentual da taxa de transferência de bits (dígitos binários) em relação a máxima taxa alcançável, como função da escolha do tamanho do prefixo cíclico percentual (em relação ao tamanho original do símbolo). Nota-se que, acima de determinado tamanho do prefixo cíclico (neste caso, aproximadamente $1\%$), o ganho de taxa de transferência de bits devido à proteção contra os fenômenos que caracterizam a distorção PCI é menor que a perda de taxa devido ao aumento da duração dos símbolos, o que explica o decaimento da taxa de bits.

\begin{figure}[h]
\centering
\includegraphics[width=1\textwidth]{Figs/taxaDeBit_vs_cp}
\caption{ Típica relação de compromisso entre o tamanho do prefixo cíclico e a taxa de transmissão de dados  \label{fig:tradeoff_cp_bitrate}}
\end{figure}
Diante desta relação de compromisso, busca-se uma solução alternativa que possa resolver a degradação do desempenho devido à distorção PCI sem comprometer a taxa de transmissão de informação: a mitigação da distorção PCI. Dentre os diversos métodos de mitigação de ISI e ICI, os quais são abordados extensivamente na literatura, o presente trabalho busca desenvolver a pré-codificação e equalização no domínio da frequência, similarmente à concepção em \cite{Malkin08}, a qual aplica pré-codificação e equalização no domínio da frequência em sistemas OFDM.

Em contraste à abordagem de \cite{Malkin08}, a qual é baseada em otimização, neste trabalho serão desenvolvidas e analisadas técnicas de pré-codificação a partir de soluções fechadas, as quais não necessitam resolver problemas de otimização para obtenção das matrizes de pré-codificação e equalização. Além disso, este trabalho difere de \cite{Malkin08} por adotar sistemas DMT e redes de acesso do tipo linha de assinante digital (DSL) como o cenário de interesse para a aplicação das técnicas de mitigação de distorção PCI desenvolvidas. Especificamente, será adotada a versão de redes DSL denominada por ``acesso rápido aos terminais do assinante'' (\emph{G.fast}) \cite{maes2012}, a qual consiste em uma versão em processo de padronização pelo Setor de Normatização das Telecomunicações (ITU-T).

Na tecnologia \emph{G.fast}, a mitigação de distorção PCI apresenta potencial para aumento do desempenho geral das linhas de assinantes contidas em um conjunto de linhas. Nestes sistemas, o operador pode ajustar o número de amostras do prefixo cíclico para lidar com a distorção do canal. No entanto, se o conjunto de linhas de transmissão apresentar linhas com respostas impulsivas de comprimentos diferentes, a linha que possuir maior resposta impulsiva irá impor um prefixo cíclico relativamente grande e, consequentemente, comprometer a taxa de transmissão de dados de todas as linhas, já que todas devem utilizar o mesmo número de amostras para o prefixo cíclico. Neste caso, técnicas que puderem adotar um prefixo cíclico reduzido com custo computacional razoável podem ser factíveis.

Com esta motivação, este trabalho tem como objetivo apresentar técnicas de mitigação de distorção PCI que superem o desempenho ótimo de sistemas DMT com prefixos cíclicos, assegurando-se de que estas não apresentem complexidades proibitivas. Sabe-se que o desempenho máximo da Figura \ref{fig:tradeoff_cp_bitrate} consiste na escolha do prefixo cíclico que melhor distribui a redução de distorção PCI e a perda de taxa de bits. No entanto, neste trabalho tem-se como objetivo apresentar técnicas que promovam, simultaneamente, a eliminação da distorção PCI e a redução da perda de taxa de bits (ao reduzir ou mesmo eliminar a utilização do prefixo cíclico). Por isso, é de se esperar que o desempenho máximo da Figura \ref{fig:tradeoff_cp_bitrate} seja superado pelas técnicas a serem desenvolvidas.

\section{Organização do Trabalho}

\begin{itemize}
\item O Capítulo \ref{sec:fundamentos_dmt} apresenta os conceitos fundamentais para o entendimento de sistemas multiportadoras e, especificamente, sistemas DMT, tais como a capacidade de transmissão de dados destes sistemas, o papel desempenhado pela convolução circular no sistema, a vantagem de se utilizar a transformada rápida de Fourier, o objetivo dos algoritmos de carregamento de bits e os cálculos necessários para controlar a potência de transmissão. Além disso, apresenta os conceitos indispensáveis ao entendimento da distorção causada por prefixo cíclico insuficiente e das técnicas de mitigação exploradas no restante do trabalho, com destaque para as análises qualitativas e quantitativas dos conceitos de interferência inter-simbólica e interferência-portadora. Ao longo do capítulo, serão introduzidas as figuras de mérito e parâmetros fundamentais à interpretação do desempenho dos sistemas simulados no restante do trabalho.\\
\item O Capítulo \ref{sec:mitigacao_pci} desenvolve técnicas de mitigação de distorção PCI através da pré-codificação de símbolos no transmissor e da equalização no receptor. São apresentadas as estruturas fundamentais de pré-codificação de ICI no transmissor, equalização de ISI no receptor e pré-codificação conjunta de ICI e ISI no transmissor, além de uma análise detalhada da complexidade destes algoritmos. Finalmente, são apresentadas as simulações de desempenho dos sistemas com a utilização das técnicas de mitigação desenvolvidas.\\
\item O Capítulo \ref{sec:power_control}
tem como objetivo introduzir os ajustes finais às técnicas de mitigação desenvolvidas no Capítulo~\ref{sec:fundamentos_dmt}, relacionados às restrições de potência e complexidade. Dependendo de características do canal, a pré-codificação desenvolvida no Capítulo \ref{sec:mitigacao_pci} pode degradar significativamente o desempenho do sistema ao tentar assegurar que os limites de potência de transmissão sejam respeitados. Assim, com o objetivo de solucionar este problema, o Capítulo \ref{sec:power_control} apresenta técnicas de normalização e ajuste da matriz de pré-codificação que asseguram, concomitantemente, a satisfação dos limites de potência de transmissão e a manutenção do desempenho ótimo do sistema. Além disso, o capítulo apresenta técnicas que promovem a redução do custo computacional da pré-codificação de símbolos, utilizando para este propósito as condições de simetria características das matrizes de pré-codificação.
\item O Capítulo \ref{ch:conclusoes} apresenta um resumo dos conceitos desenvolvidos e dos resultados obtidos no trabalho e uma indicação para estudos futuros relacionados ao tema.
\item O Apêndice \ref{app:ortogonalidade} apresenta noções a respeito da quebra da ortogonalidade de portadoras em sistemas DMT.
%\item O Apêndice \ref{app:isi_ici_gaussiana} apresenta uma justificativa empírica para a suposição da distribuição aleatória das interferência inter-simbólica e inter-portadora.
\item O Apêndice \ref{app:cond_prec_matrix} analisa a simetria existente na matriz de pré-codificação e evidencia o motivo pelo qual os símbolos pré-codificados mantêm a mesma simetria dos símbolos originais (simetria Hermitiana).
\end{itemize}


