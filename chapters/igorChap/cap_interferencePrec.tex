\chapter{Mitigação de Distorção PCI}
\label{sec:mitigacao_pci}

\section{Introdução}

Na seção anterior, apresentou-se o fato de que, devido a um prefixo cíclico insuficiente, surge distorção na saída do canal na forma de interferência inter-simbólica e inter-portadora, as quais foram denominadas como distorção por prefixo cíclico insuficiente (distorção PCI). Esta distorção pode impactar severamente no desempenho de um sistema DMT, motivo pelo qual técnicas que promovam a sua mitigação são extensamente abordadas na literatura.

A grande relação envolvida na mitigação de distorção PCI em sistemas DMT é o compromisso entre o comprimento do prefixo cíclico e a perda que este provoca na taxa de informação. Sabe-se que, para evitar a ocorrência de distorção PCI, o prefixo cíclico deve ser suficiente para cobrir a dispersão causada pelo canal. Por outro lado, sabe-se que o prefixo cíclico diminui a taxa de transmissão de símbolos DMT, já que os símbolos deixam de ter $N$ amostras para ter $\nu + N$ amostras quando o prefixo é adicionado. Assim, ao aumentar o prefixo cíclico, reduz-se a distorção PCI com a desvantagem da diminuição da taxa de transmissão de símbolos. Analogamente, ao diminuir o prefixo cíclico, aumenta-se a taxa de transmissão de símbolos com a desvantagem do aumento da distorção PCI.

Tipicamente, as respostas impulsivas dos canais utilizados em sistemas DMT apresentam grande número de amostras não nulas, devido às elevadas frequências de amostragem. Diante disto, não é prático utilizar valores de prefixo cíclico suficientes para cobrir a dispersão do canal, já que estes reduziriam a taxa de transmissão de símbolos a valores muito baixos. Assim, comumente os sistemas DMT operam com prefixos cíclicos insuficientes, os quais são projetados para cobrir a parte mais significativa da energia proveniente das interferências inter-portadoras e inter-simbólica. Ainda assim, verifica-se que a distorção PCI remanescente degrada significativamente o desempenho do sistema.

A degradação do desempenho e a possibilidade de aumento de taxa de informação com a mitigação de distorção PCI é a motivação para seu estudo. Diversas métodos são conhecidos para este propósito, destacando-se a equalização no domínio do tempo (TEQs) ou encurtamento de canal \cite{chanshort2001,channelshort2003}, equalizadores no domínio da frequência no receptor \cite{Trautmann2002, Park04} e a pré-codificação de símbolos \cite{Malkin08,Cheong98}. Além destas, técnicas como algoritmos iterativos de estimação e mitigação de interferência \cite{stuber1998}, algoritmos de mitigação a partir da exploração de subcanais não utilizados \cite{huang2009}, algoritmos de reserva de subcanais \cite{malkin2007}, dentre outros. Estes métodos, sejam eles baseados em processamento no receptor, no transmissor ou em ambos, introduzem complexidade no sistema, a qual deve ser mantida em um nível aceitável para não contrariar uma das grandes vantagens oferecida por sistemas multiportadora, a baixa complexidade.

Neste trabalho, adota-se a tecnologia de acesso denominada por ``acesso rápido aos terminais do assinante'' (G.fast) \cite{maes2012} como o cenário de interesse para a aplicação das técnicas de mitigação de distorção PCI desenvolvidas. Esta consiste em uma versão de redes DSL em processo de padronização pelo Setor de Normatização das Telecomunicações (ITU-T), a qual busca satisfazer as demandas de altas taxas de bit (até $1$~Gb/s) características dos serviços da quarta geração de banda larga (4GBB) \cite{tno_gfast} utilizando ainda as instalações de cobre (telefônicas) existentes nos domicílios. Para tanto, esta tecnologia adota a solução denominada fibra até a residência (FTTH), na qual procura-se levar a fibra ótica às proximidades das residências, reduzindo o comprimento dos enlaces de cobre (par-trançado) que levam a transmissão do ponto mais próximo onde há fibra até o interior das instalações do usuário.

Nesta tecnologia, a mitigação de distorção PCI apresenta potencial para aumento do desempenho geral das linhas de assinantes contidas em um conjunto de linhas (\textsl{binder}). Nestes sistemas, o operador pode ajustar o comprimento do prefixo cíclico para lidar com a distorção do canal. No entanto, se o \textsl{binder} apresentar linhas com comprimentos diferentes, a linha que possuir maior resposta impulsiva irá impor um prefixo cíclico relativamente grande e, consequentemente, diminuir a taxa de símbolos de todas as linhas, já que, devido a duplexação por divisão de tempo sincronizada (S-TDD), todas devem utilizar o mesmo prefixo cíclico. Neste caso, técnicas que puderem adotar um prefixo cíclico reduzido com custo computacional razoável podem ser factíveis.

Neste trabalho, tem-se como objetivo analisar técnicas de mitigação de distorção PCI em sistemas DMT através da pré-codificação de símbolos no transmissor e equalização no receptor. Estas são extensivamente abordadas para redes sem-fio com sistemas OFDM em \cite{malkin2009}, no qual utiliza-se otimização convexa para pré-codificar os símbolos visando obter a máxima taxa de bits e satisfazer as máximas alocações de potência para cada subcanal. Em conjunto com a pré-codificação, em \cite{malkin2009} é desenvolvido um algoritmo que toma vantagem da alta correlação que existe entre as interferências inter-portadoras e inter-simbólicas no domínio da frequência causadas por tons consecutivos, o que proporciona a diminuição do número de operações necessárias para a pré-codificação. Adicionalmente,  utiliza-se cancelamento de interferência sucessiva (SIC) no receptor para atuar em conjunto com a pré-codificação na mitigação de ICI, o que proporciona ainda menor complexidade.

Em contraste à abordagem de \cite{malkin2009}, a qual é baseada em otimização, neste trabalho serão analisadas técnicas de pré-codificação a partir de soluções fechadas, as quais não necessitam de longas fases de treinamento dedicadas à solução do problema de otimização\footnote{O problema de otimização referido consiste em calcular a matriz de pré-codificação ótima para a maximização da taxa de bits, tendo como restrição os limites de potência.}. Nesta capítulo, serão apresentadas as estruturas fundamentais de pré-codificação de ICI no transmissor (Seção \ref{sec:prec_ICI}), equalização de ISI no receptor (Seção \ref{sec:isi_equalization}) e pré-codificação conjunta de ICI e ISI no transmissor (Seção \ref{sec:precodificacao_isi_ici}). Sequencialmente, apresenta-se na Seção \ref{sec:complexidade_precodificacao} uma análise da complexidade dos algoritmos desenvolvidos. Finalmente, simula-se o desempenho de sistemas com a utilização da pré-codificação e equalização na Seção \ref{sec:ici_isi_pre_sim}. 

É importante observar que, no desenvolvimento apresentado, a complexidade dos algoritmos  será relativamente alta e a alteração de potência no símbolo introduzida pelo processo de pré-codificação será solucionada de maneira pouco elaborada, já que as alternativas para redução de complexidade e controle de potência serão apresentadas em detalhes no Capítulo \ref{sec:power_control}. 

\section{Pré-codificação de Símbolos para Mitigar ICI}
\label{sec:prec_ICI}

Inicialmente, será considerada a técnica de pré-codificação de símbolos para mitigação de interferência inter-portadora (ICI) no símbolo recebido. Esta técnica consiste em introduzir pré-distorção no símbolo a ser transmitido, de maneira que, quando efetivamente transmitido, esta seja eliminada pela distorção natural do canal, restando ao receptor o símbolo desejado sem interferência inter-portadora.

\begin{figure}[htbp]
\centering
\includegraphics[width=1\textwidth]{./figs/cadeia_dmt_precoder}
\caption{Cadeia DMT com a inclusção do pré-codificador.
\label{fig:cadeia_dmt_prec}}
\end{figure}

Par tanto, introduz-se um bloco pré-codificador de símbolos na cadeia DMT original (Figura \ref{fig:cadeia_ofdm}), o qual tem como finalidade transformar o símbolo DMT original, obtido a partir do mapeador QAM, em um símbolo pré-codificado. Neste caso, devido ao fato de que o pré-codificador tem como entrada o símbolo DMT no domínio da frequência, diz-se que o pré-codificador atua no domínio da frequência. A cadeia DMT resultante está apresentada na Figura \ref{fig:cadeia_dmt_prec}, na qual destaca-se o pré-codificador entre o mapeador de constelações e o operador IFFT. 

O pré-codificador consiste em uma matriz $\mathbf{W}$, a qual multiplica o símbolo DMT (vetor $\mathbf{X}$), resultando no símbolo pré-codificado $\tilde{\mathbf{X}}$. Assim, visando a chegada do símbolo original ($\mathbf{X}$) no receptor, transmite-se o símbolo pré-codificado $\tilde{\mathbf{X}}$, como será visto no desenvolvimento que segue. Os vetores de entrada e saída do bloco pré-codificador estão destacados no esquema da Figura \ref{fig:pre_codificador}. 

\begin{figure}[htbp]
\centering
\tikzstyle{box} = [rectangle, draw, text width=5em, text centered, minimum height=5em]
\tikzstyle{boxs} = [rectangle, draw, text width=5em, text centered, minimum height=5em, color=red]
\tikzstyle{line} = [draw, -latex']
\begin{tikzpicture}[node distance = 2cm, auto]
\node [] (DMTs) [] {\text{\parbox{3 cm}{\centering Símbolo DMT Transmitido}}};
\node [box] (Prec) [right of=DMTs,xshift = 8em] {Pré-codificador};
\node [] (IFFT) [right of=Prec, xshift=8em] {\text{\parbox{3 cm}{\centering Símbolo pré-codificado}}};
\path [line] (DMTs) -- node[] {$\mathbf{X}$}  (Prec);
\path [line] (Prec) -- node[] {$\tilde{\mathbf{X}} = \mathbf{WX}$} (IFFT);
\end{tikzpicture}
\caption{Esquema de pré-codificação.
\label{fig:pre_codificador}}
\end{figure}

É importante destacar que a matriz de pré-codificação $\mathbf{W}$ é função do canal, isto é, necessita da resposta impulsiva do canal e do conhecimento do atraso ($n_0$) para ser construída. Isto é esperado, já que o pré-codificador precisa conhecer a distorção introduzida pelo canal para poder promover a sua mitigação. Assim, é necessário que o transmissor tenha a informação do canal para realizar a pré-codificação, o que torna necessário a estimação de canal.

Por ora, assume-se que o transmissor detém perfeito conhecimento do canal. Na prática, este conhecimento perfeito não é facilmente garantido \cite{malkin2009}. No entanto, em sistemas de comunicação no qual o canal não apresenta variações rápidas, como é o caso de sistemas de comunicação cabeados, é possível obter a estimativa do canal com maior confiabilidade. 

\subsection{Modelamento do Símbolo Recebido no Domínio da Frequência}

Para iniciar o desenvolvimento do pré-codificador, tem-se como objetivo conhecer a transformada discreta de Fourier do conteúdo de ISI e ICI no domínio do tempo introduzido pelo símbolo transmitido. Em outras palavras, deseja-se expressar a DFT da ISI ($\mathbf{Y}_\text{ISI}$) e a DFT da ICI ($\mathbf{Y}_\text{ICI}$) conforme as expressões abaixo:
\begin{align}
\mathbf{Y}_\text{ISI}^{(i)} &=\mathbf{ T_\text{ISI}}\mathbf{X^{(i-1)}} \label{eq:ISI_Tisi}\\
\mathbf{Y}_\text{ICI}^{(i)} &=\mathbf{ T_\text{ICI}\mathbf{X^{(i)}}} \label{eq:ISI_Tici}
\end{align}
onde $\mathbf{ T_\text{ISI}}$ e $\mathbf{ T_\text{ICI}}$ são as matrizes que transformam linearmente os símbolos no domínio da frequência $\mathbf{X}$ em suas ISIs e ICIs no domínio da frequência. 

Para expressar dessa maneira, utiliza-se as representações matriciais da ISI e ICI no domínio do tempo, apresentadas nas Equações \ref{eq:ISI_matrix_rep} e \ref{eq:ICI_matrix_rep_vec}, as quais são repetidas abaixo por conveniência:
\begin{align}
\mathbf{y}_\text{ISI}^{(i)} &= \mathbf{ H_\text{ISI}} \mathbf{x}^{(i-1)}[((n + n_0))_N] \nonumber\\
\mathbf{y}_\text{ICI}^{(i)} &= \mathbf{ H_\text{ICI}} \mathbf{x}^{(i)}[((n + n_0))_N] \nonumber
\end{align}
onde $((n + n_0))_N$ denota o deslocamento circular de $n_0$ amostras dentre as $N$ amostras do símbolo.

Em seguida, calcula-se as DFTs destes vetores de ISI e ICI segundo a definição da Equação \ref{eq:m_dft}:
\begin{align}
\mathbf{Y}_\text{ISI}^{(i)} &= \mathbf{Q} \mathbf{ H_\text{ISI}} \mathbf{x}^{(i-1)}[((n + n_0))_N]
\label{eq:partial_ISI_DFT}\\
\mathbf{Y}_\text{ICI}^{(i)} &= \mathbf{Q} \mathbf{ H_\text{ICI}} \mathbf{x}^{(i)}[((n + n_0))_N] 
\label{eq:partial_ICI_DFT}
\end{align}

Adicionalmente, observa-se que os símbolos no domínio do tempo com deslocamento circular podem ser expressos como a IDFT de suas representações no domínio da frequência, isto é, a IDFT de suas DFTs. Para tanto, considera-se a propriedade do deslocamento circular no domínio do tempo correspondente a multiplicação dos subsímbolos componentes do vetor DFT por exponenciais complexas, conforme abaixo \cite{dspambardar}:

$$ \mathbf{x}[((n + n_0))_N] \leftrightarrow X_ke^{j2\pi k n_0/N} $$
onde $\leftrightarrow$ denota a correspondência entre o domínio do tempo (lado esquerdo) e o domínio da frequência (lado direito).


Portanto, utilizando a definição de IDFT da Equação \ref{eq:m_idft}, expressa-se o vetor $ \mathbf{x}[((n + n_0))_N]$ como:
\begin{align}
\mathbf{x}^{(i)}[((n + n_0))_N] &=  \mathbf{Q}^\text{H} \mathbf{\Theta_{n_0} \mathbf{X^{(i)}}}  
\label{eq:x_circ_shift_time}
\end{align}
onde $\mathbf{\Theta}_{n_0}$ é uma matriz diagonal, função do atraso $n_0$, cujos elementos correspondem ao deslocamento de fase em cada subcanal, originados pelo deslocamento circular do símbolo no domínio do tempo. Os elementos na diagonal principal em cada linha $k$ da matriz são as exponenciais complexas $e^{j2\pi k n_0/N}$, isto é, $\mathbf{\Theta}(n_0)$ é dada por:
\begin{align}
\mathbf{\Theta(n_0)} = \left[ \begin{array}{ccccc}
1 & 0 & 0 & \cdots & 0\\
0 & e^{-j\frac{2\pi n_0}{N}} & 0 & \cdots & 0\\
\vdots & \vdots & \ddots & \vdots & \vdots\\
0 & 0 & 0 & e^{-j\frac{2\pi n_0}{N}(N-2)} & 0\\
0 & 0 & 0 & \cdots & e^{-j\frac{2\pi n_0}{N}(N-1)}\\
\end{array} \right]
\end{align}

Assim, substituindo a Equação \ref{eq:x_circ_shift_time} nas Equações \ref{eq:partial_ISI_DFT} e \ref{eq:partial_ICI_DFT}, resulta em:
\begin{align}
\mathbf{Y_\text{ISI}^{(i)}} &=  \mathbf{Q} \mathbf{ H_\text{ISI}} \mathbf{Q}^\text{H} \mathbf{\Theta}_{n_0} \mathbf{X}^{(i-1)}
\label{eq:partial_ISI_DFT2}\\
\mathbf{Y_\text{ICI}^{(i)}} &=  \mathbf{Q} \mathbf{ H_\text{ISI}} \mathbf{Q}^\text{H} \mathbf{\Theta_{n_0} \mathbf{X^{(i)}}}
\label{eq:partial_ICI_DFT2}
\end{align}

Portanto, conclui-se que as matrizes $\mathbf{ T_\text{ISI}}$ e $\mathbf{ T_\text{ICI}}$ das Equações \ref{eq:ISI_Tisi} e \ref{eq:ISI_Tici} são dadas, respectivamente, por:
\begin{align}
\mathbf{ T_\text{ISI}} &= \mathbf{Q} \mathbf{ H_\text{ISI}} \mathbf{Q}^\text{H} \mathbf{\Theta_{n_0}}\\
\mathbf{ T_\text{ICI}} &= \mathbf{Q} \mathbf{ H_\text{ICI}} \mathbf{Q}^\text{H} \mathbf{\Theta_{n_0}}
\label{eq:tici_definition}
\end{align}

Na sequência, similarmente à Equação \ref{eq:yideal_yisi_ici}, divide-se o símbolo recebido no domínio da frequência em três parcelas: a parte ideal, a qual seria recebida se não houvesse distorção PCI; a parte correspondente à ISI e a parte correspondente à ICI.
\begin{align}
\mathbf{Y}^{(i)}= \mathbf{Y}_\text{Ideal}^{(i)} + \mathbf{Y}_\text{ISI}^{(i)} + \mathbf{Y}_\text{ICI}^{(i)}
\label{eq:rx_symbol_freq_domain}
\end{align}

Substituindo as Equações \ref{eq:ISI_Tisi} e  \ref{eq:ISI_Tici} na equação acima, resulta na seguinte representação para o $i$-ésimo símbolo recebido no domínio da frequência:
\begin{align}
\mathbf{Y}^{(i)} &=  \mathbf{Y}_\text{Ideal}^{(i)} + \mathbf{ T_\text{ISI}} \mathbf{X^{(i-1)}} + \mathbf{ T_\text{ICI} \mathbf{X^{(i)}}}\nonumber
\end{align}
onde a parcela $\mathbf{Y}_\text{Ideal}^{(i)}$ é obtida a partir da DFT da Equação \ref{eq:y_ideal}, conforme abaixo:
\begin{align}
\mathbf{Y}_\text{Ideal}^{(i)} = \mathbf{Q} \mathbf{y}_\text{Ideal}^{(i)}  = \mathbf{Q} \mathbf{ \tilde{H} } \mathbf{x}^{(i)}[((n + n_0))_N] = \mathbf{Q} \mathbf{ \tilde{H} } \mathbf{Q}^\text{H} \mathbf{\Theta_{n_0} \mathbf{X^{(i)}}}
\label{eq:y_ideal_freq_domain}  
\end{align}
Pode-se demonstrar que os vetores base da matriz DFT diagonalizam a matriz circulante \cite{malkin2009}, já que os autovetores da matriz circulante correspondem aos vetores linearmente independentes que formam as bases da matriz DFT. Portanto, a decomposição em autovalores da matriz circulante $\tilde{\mathbf{H}}$ resulta em:
\begin{align}
\tilde{\mathbf{H}} =  \mathbf{Q}^H \tilde{\mathbf{\Lambda}} \mathbf{Q}
\label{eq:hcirc_diag}
\end{align}
onde $\tilde{\mathbf{\Lambda}}$ é uma matriz diagonal cujos elementos da diagonal principal correspondem à DFT da resposta impulsiva do canal e são os autovalores de $\tilde{\mathbf{H}}$ \cite{gray2006}.

Isolando a matriz de ganhos do canal $\tilde{\mathbf{\Lambda}}$ na Equação \ref{eq:hcirc_diag} e considerando que $\mathbf{Q}$ é uma matriz unitária ($\mathbf{Q}\mathbf{Q}^H=1$), obtém-se:
\begin{align}
\tilde{\mathbf{\Lambda}} = \mathbf{Q}\tilde{\mathbf{H}}Q^H 
\label{eq:h_gain_diag}
\end{align}

Substituindo a Equação \ref{eq:h_gain_diag} na Equação \ref{eq:y_ideal_freq_domain}, resulta em:
\begin{align}
\mathbf{Y}_\text{Ideal}^{(i)} = \tilde{\mathbf{\Lambda}} \mathbf{\Theta_{n_0} \mathbf{X^{(i)}}}
\end{align}

Pode-se, no entanto, agrupar o produto $\tilde{\mathbf{\Lambda}}\mathbf{\Theta}_{n_0}$ em somente uma matriz, a qual será denominada $\mathbf{\Lambda}$:
\begin{align}
\mathbf{\Lambda}  = \tilde{\mathbf{\Lambda}}\mathbf{\Theta}_{n_0} = \mathbf{Q}\tilde{\mathbf{H}}\mathbf{Q}^H\mathbf{\Theta}_{n_0}
\label{eq:lambda_definition}
\end{align}

Neste caso, o vetor $\mathbf{Y}_\text{Ideal}^{(i)}$ pode ser expresso como:
\begin{align}
\mathbf{Y}_\text{Ideal}^{(i)} = \mathbf{\Lambda \mathbf{X^{(i)}}}
\end{align}

Finalmente, substituindo a expressão acima na Equação \ref{eq:rx_symbol_freq_domain}, obtém-se a expressão que modela o símbolo recebido no domínio da frequência em função dos símbolos presente e precedente:
\begin{align}
\mathbf{Y}^{(i)} &=\left(\mathbf{\Lambda} + \mathbf{ T_\text{ICI}}\right) \mathbf{X^{(i)}} + \mathbf{ T_\text{ISI}} \mathbf{X^{(i-1)}}\nonumber\\
\mathbf{Y}^{(i)} &= \mathbf{A} \mathbf{X^{(i)}} + \mathbf{ T_\text{ISI}} \mathbf{X^{(i-1)}}
\label{eq:model_y_rx_freq_domain}
\end{align}
onde $\mathbf{A}=\left(\mathbf{\Lambda} + \mathbf{ T_\text{ICI}}\right)$, isto é, $\mathbf{A}$ é a matriz com os ganhos dos subcanais diretos e os ganhos de \textsl{crosstalk} entre os subcanais (devido à ICI).

No entanto, ao considerar o pré-codificador para mitigar somente a ICI, assume-se que a ISI seja completamente eliminada no receptor, como em  \cite{Malkin08,Park04}. Por isso, pode-se modelar o símbolo recebido como:
\begin{align}
\mathbf{Y}^{(i)} &= \mathbf{A} \mathbf{X^{(i)}}
\label{eq:model_y_rx_freq_domain_ici}
\end{align}

\subsection{Pré-codificação de ICI}

A partir da expressão do símbolo recebido afetado por distorção PCI no domínio da frequência, apresentada na Equação \ref{eq:model_y_rx_freq_domain_ici}, é direto demonstrar que é possível escolher o símbolo a ser transmitido $\tilde{X}$ de maneira que o símbolo recebido seja o símbolo ideal, isto é:
\begin{align}
\mathbf{Y}^{(i)} &= \mathbf{Y}_\text{Ideal}^{(i)} = \mathbf{\Lambda} \mathbf{X^{(i)}} \nonumber
\end{align}

Para tanto, é necessário que o símbolo pré-codificado seja dado por:
\begin{align}
\mathbf{\tilde{X}}^{(i)} = \mathbf{A}^{-1}\mathbf{\Lambda} \mathbf{X^{(i)}}
\label{eq:precoder_ici_isi}
\end{align}
na qual assume-se que a matriz $\mathbf{A}$ seja invertível.

Neste caso, como o símbolo pré-codificado é o produto entre a matriz pré-codificadora ($\mathbf{W}$) e o símbolo original ($\mathbf{X}$), conclui-se que a matriz pré-codificadora é dada por:
\begin{align}
\mathbf{W} =  \mathbf{A}^{-1}\mathbf{\Lambda}
\label{eq:prec_no_norm}
\end{align}

\subsection{Considerações sobre Potência}

Essencialmente, este pré-codificador força o canal real a não apresentar \textsl{crosstalk} entre os subcanais, ao providenciar a diagonalização da matriz do canal real ($\mathbf{A}$). Este pré-codificador é similar a estrutura de diagonalização do canal apresentada para eliminação de \textsl{crosstalk} entre as linhas de transmissão em \cite{Cendrillon07}, na qual, ao invés de eliminar o \textsl{crosstalk} entre os subcanais (ICI), elimina-se o \textsl{crosstalk} entre os usuários cujos canais estão em um mesmo \textsl{binder}. 

Contudo, a eliminação de \textsl{crosstalk} entre os subcanais pode demandar aumentos de potência de transmissão, como um preço a ser pago para forçar a diagonalização do canal. A operação de pré-codificação altera a distribuição de potência do símbolo transmitido, podendo esta ultrapassar os limites de potência estabelecidos para cada subcanal (máscara de potência). Para visualizar este efeito, observa-se que o $k$-ésimo subsímbolo do símbolo DMT pré-codificado é dado por:
\begin{align}
\tilde{X}_k &= \mathbf{w}_k \mathbf{X}\nonumber\\
&= \sum \limits_{m=0}^{N-1} w_{k,m} X_m
\label{eq:xtilde_innerp}
\end{align}
onde o $\mathbf{w}_k $ é o vetor-linha com os elementos da $k$-ésima linha de $\mathbf{W}$ e $w_{k,m}$ é o elemento na $k$-ésima linha e $m$-ésima coluna de $\mathbf{W}$, isto é, $\mathbf{w}_k  = \left[ w_{k,0}, w_{k,1}, \cdots, w_{k,N-1} \right]$.

Sequencialmente, nota-se que a densidade espectral de potência média ($\tilde{s}_k$) no $k$-ésimo subcanal pré-codificado é dada por\footnote{A densidade espectral de potência apresentada corresponde à definição de DFT adotada na Equação \ref{eq:q_matrix}.}:
\begin{align}
\tilde{s}_k \stackrel{\Delta}{=} \frac{E \{ \left| \tilde{X}_k \right|^2 \}}{N}
\label{eq:psd_definition}
\end{align}
onde o operador $E\{ \cdot \}$ denota o valor esperado.

Substituindo $\tilde{X}_k$ da Equação \ref{eq:xtilde_innerp}, obtém-se:
\begin{align}
\tilde{s}_k &=  \frac{E \left\{ \left| \sum \limits_{m=0}^{N-1} w_{k,m} X_m \right|^2 \right\} }{N}
\end{align}

Contudo, devido ao valor absoluto de um somatório ser menor ou igual o somatório dos valores absolutos (propriedade da desigualdade triangular), a PSD no subcanal $k$ torna-se:
\begin{align}
\tilde{s}_k &\leq  \frac{E \left\{  \sum \limits_{m=0}^{N-1} \left|w_{k,m} X_m \right|^2 \right\} }{N} \nonumber\\
&\leq  \frac{E \left\{  \sum \limits_{m=0}^{N-1} \left|w_{k,m} \right|^2 \left|X_m \right|^2 \right\} }{N} \nonumber
\end{align}

Adicionalmente, dado que a matriz de pré-codificação $\mathbf{W}$ não é aleatória, o valor esperado aplica-se somente ao termo $X_m$. Neste caso, a expressão torna-se:
\begin{align}
\tilde{s}_k  &\leq  \frac{\sum \limits_{m=0}^{N-1} \left|w_{k,m} \right|^2 E \left\{  \left|X_m \right|^2 \right\} }{N} \nonumber
\end{align}

Substituindo a definição de PSD da Equação \ref{eq:psd_definition} no termo $\frac{E \left\{  \left|X_m \right|^2 \right\} }{N}$, tem-se:
\begin{align}
\tilde{s}_k  &\leq  \sum \limits_{m=0}^{N-1} \left|w_{k,m} \right|^2 s_m \nonumber
\end{align}

Finalmente, considera-se que a PSD é distribuída igualmente entre os subcanais, o que implica que o termo correspondente a PSD do subsímbolo original no $m$-ésimo subcanal $s_m$ é constante ($s_m = s$). Assim, a expressão torna-se:
\begin{align}
\tilde{s}_k  &\leq  s_k \sum \limits_{m=0}^{N-1} \left|w_{k,m} \right|^2  \nonumber\\
&\leq  s_k \| \mathbf{w}_k \|^2 
\label{eq:psd_bound}
\end{align}

Portanto, vê-se que a PSD no subcanal $k$ devido à transmissão de símbolos pré-codificados é menor ou igual a PSD original (correspondente à transmissão de símbolos não pré-codificados) multiplicada pela norma Euclidiana da $k$-ésima linha da matriz de pré-codificação ($\| \mathbf{w}_k \|$). Portanto, se $\| \mathbf{w}_k \|$ for maior que 1, a PSD dos símbolos pré-codificados pode ultrapassar a PSD original, isto é, $\tilde{s}_k$ pode ser maior que $s_k$. 

Neste caso, a matriz de pré-codificação é normalizada, visando garantir que a PSD do símbolo pré-codificado seja sempre menor que a PSD do símbolo original. Para tanto, introduz-se um escalar $\beta$ na matriz de pré-codificação da Equação \ref{eq:prec_no_norm}, resultado na matriz de pré-codificação normalizada ($\mathbf{W}'$):
\begin{align}
\mathbf{W}' =  \frac{1}{\beta}\mathbf{A}^{-1}\mathbf{\Lambda}
\label{eq:matriz_precoder_beta}
\end{align}

Neste caso, a PSD do $k$-ésimo símbolo pré-codificado passa a ser dada por: 
\begin{align}
\tilde{s}_k &\leq \frac{ \| \mathbf{w}_k \|^2 }{\beta^2}  s_k
\label{eq:psd_bound_norm}
\end{align}

Portanto, para garantir que todos os subsímbolos apresentem PSD abaixo da PSD original, $\beta$ é escolhido como a máxima norma Euclidiana dentre as normas das linhas da matriz $\mathbf{W}$:
\begin{align}
\beta = \max \left\{ \| \mathbf{w}_k \| \right\}
\label{eq:fator_normalizacao}
\end{align}

\section{Equalização de ISI no Receptor}
\label{sec:isi_equalization}

A interferência inter-simbólica (pós-cursora) que afeta o símbolo recebido é, por definição, função somente do canal e do símbolo anterior. Assim, considerando que o receptor conheça o canal e a maneira como este transforma o símbolo anterior em ISI no símbolo atual, é possível que este estime a ISI que afeta o símbolo recebido. Isto é possível devido ao fato de que, no momento do processamento do símbolo atual, o símbolo anterior já é conhecido pelo receptor, pois este foi detectado anteriormente.

Considerando novamente o símbolo $\mathbf{Y}^{(i)}$ recebido como
\begin{align}
\mathbf{Y}^{(i)} &= \mathbf{A} \mathbf{X^{(i)}} + \mathbf{ T_\text{ISI}} \mathbf{X^{(i-1)}} \nonumber
\end{align}
e assumindo que o receptor conheça $\mathbf{X^{(i-1)}}$, conclui-se que é possível mitigar a ISI a partir da eliminação da parcela correspondente a esta, isto é, a eliminação (subtração) de $\mathbf{ T_\text{ISI}} \mathbf{X^{(i-1)}}$.

No entanto, esta implementação pode falhar na perfeita mitigação de ISI, já que o símbolo anterior utilizado para remoção é, na realidade, o símbolo detectado, o qual pode apresentar erros de detecção. Consequentemente, a equalização de ISI está sujeita à propagação de erro, já que o erro na detecção do símbolo implica no erro na remoção de ISI, que por sua vez afeta a probabilidade de erro na detecção do símbolo, e assim por diante. 

Ainda assim, considerando que os sistemas de comunicação são projetados para baixa probabilidade erro, espera-se que o efeito da propagação de erro seja insignificante para o desempenho do sistema. Este será analisado nas simulações apresentadas na Seção \ref{sec:ici_isi_pre_sim}.

\section{Pré-codificação de Símbolos para Mitigação de ICI e ISI}
\label{sec:precodificacao_isi_ici}

A mitigação conjunta de ICI e ISI a partir de pré-codificação de símbolos no transmissor também é possível. Assim, como alternativa à utilização do esquema de pré-codificação de ICI no transmissor e equalização de ISI no receptor, pode-se concentrar no transmissor a complexidade introduzida no sistema.

A formulação da matriz de pré-codificação de ICI e ISI tem um desenvolvimento análogo ao utilizado para a matriz pré-codificadora de ICI. Inicialmente, substitui-se a expressão para o $i$-ésimo símbolo recebido no domínio da frequência ($\mathbf{Y}^{(i)}$), dada pela Equação \ref{eq:model_y_rx_freq_domain}, na condição de que o símbolo real recebido seja o símbolo ideal ($\mathbf{Y}^{(i)} =\mathbf{\Lambda} \mathbf{X^{(i)}}$). A substituição resulta em: 
\begin{align}
\mathbf{A} \mathbf{\tilde{X}}^{(i)} + \mathbf{ T_\text{ISI}} \mathbf{\tilde{X}}^{(i-1)} = \mathbf{\Lambda} \mathbf{X^{(i)}}\nonumber
\end{align}
na qual é importante observar que o símbolo DMT precedente em questão é o símbolo pré-codificado, já que este é o símbolo efetivamente transmitido.

Isolando o símbolo DMT pré-codificado de referência ($\mathbf{\tilde{X}}^{(i)}$) na equação, obtém-se:
\begin{align}
\mathbf{\tilde{X}}^{(i)} = \mathbf{A}^{-1} \left( \mathbf{\Lambda} \mathbf{X^{(i)}} - \mathbf{ T_\text{ISI}} \mathbf{\tilde{X}}^{(i-1)} \right)
\label{eq:precoder_ici_isi}
\end{align}
na qual assume-se que a matriz $\mathbf{A}$ seja invertível.

Neste caso, utilizando a definição da matriz pré-codificadora de ICI ($\mathbf{W}$) em
\ref{eq:prec_no_norm}, pode-se expressar o símbolo pré-codificado a partir de duas matrizes pré-codificadoras:
\begin{align}
\mathbf{\tilde{X}}^{(i)} = \mathbf{W} \mathbf{X^{(i)}} - \mathbf{W}_2 \mathbf{\tilde{X}}^{(i-1)} 
\end{align}
onde $\mathbf{W}_2 = \mathbf{A}^{-1}\mathbf{ T_\text{ISI}}$.

Finalmente, pode-se ainda incluir o fator de normalização na matriz pré-codificadora de ICI, visando satisfazer os limites de potência sob o ponto de vista da parcela $\mathbf{W} \mathbf{X^{(i)}}$. Neste caso, utiliza-se a definição da matriz pré-codificadora normalizada ($\mathbf{W}'$) da Equação~\ref{eq:matriz_precoder_beta}, resultando no seguinte símbolo pré-codificado:
\begin{align}
\mathbf{\tilde{X}}^{(i)} = \mathbf{W}' \mathbf{X^{(i)}} - \mathbf{W}_2 \mathbf{\tilde{X}}^{(i-1)} 
\label{eq:precoder_ici_isi}
\end{align}

\section{Complexidade dos Algoritmos de pré-codificação}
\label{sec:complexidade_precodificacao}

Até o momento, apresentou-se a pré-codificação cuja complexidade será denominada complexidade completa, na qual a matriz pré-codificadora $\mathbf{W}$ (ou $\mathbf{W}'$)  consiste em uma matriz complexa de dimensões $N \times N$. Para esta matriz, a pré-codificação dos símbolos demanda $N^2$ multiplicações escalares complexas e $N(N-1)$ adições escalares complexas, similarmente ao cálculo direto da DFT e IDFT, como apresentado na Seção \ref{sec:complexidade_dft}.

Dado que uma multiplicação escalar complexa requer 4 multiplicações escalares reais e 2 adições escalares reais, enquanto uma adição escalar complexa requer 2 adições escalares reais, o algoritmo requer um total de $4N^2$ multiplicações escalares reais e  $4N^2 - 2N$ adições escalares reais. Este número de operações torna-se bastante elevado à medida que $N$ torna-se considerável. 

Similarmente, o algoritmo de equalização de ISI no receptor requer a multiplicação de uma matriz complexa $N \times N$ ($\mathbf{T}_\text{ISI}$) por um vetor-coluna complexo de $N$ elementos, além de $N$ adições escalares complexas para subtrair o resultado da multiplicação ($\mathbf{T}_\text{ISI} \mathbf{\tilde{X}}^{i-1}$) do símbolo recebido. Portanto, a equalização demanda $4N^2$ multiplicações escalares reais e $4N^2$ adições escalares reais.  

Enquanto isso, a pré-codificação conjunta de ICI e ISI da Equação \ref{eq:precoder_ici_isi} requer 2 multiplicações de matrizes complexas de dimensões $N \times N$ ($\mathbf{W}$ e $\mathbf{W}_2$) por vetores-coluna complexos de N elementos, além de $N$ adições escalares complexas para somar as parcelas. Portanto, o algoritmo necessita realizar $8N^2$ multiplicações escalares reais e $8N^2 - 2N$ adições escalares reais no transmissor.

\begin{table}[htbp]
\centering
\begin{tabular}{ p{5cm} | p{2.5 cm} | p{2.5cm} | p{2.5cm} | p{2.5cm} |}
 & \multicolumn{2}{|c|}{Transmissor} & \multicolumn{2}{|c|}{Receptor}\\
\hline
Algoritmo & Multiplicações escalares reais & Adições escalares reais & Multiplicações escalares reais & Adições escalares reais\\
\hline
Pré-codificação de ICI + Equalização de ISI & $4N^2$ & $4N^2 - 2N$ & $4N^2$ & $4N^2$\\
\hline
Pré-codificação de ICI e ISI & $8N^2$ & $8N^2 - 2N$ & - & - \\
\hline
\end{tabular}
\caption{Complexidade dos algoritmos de pré-codificação. \label{tb:complexidade_dp}}
\end{table}

A complexidade introduzida no transmissor e no receptor pelos algoritmos apresentados está resumida na Tabela \ref{tb:complexidade_dp}. Vê-se que a escolha entre os dois esquemas de mitigação baseia-se na necessidade de distribuir o aumento da complexidade entre transmissor e receptor, ou não. Na seção \ref{sec:power_control}, serão apresentadas alternativas para redução da complexidade destes algoritmos.

\section{Simulações}
\label{sec:ici_isi_pre_sim}

Apresenta-se a simulação dos métodos de mitigação de distorção PCI desenvolvidos nas seções anteriores. Para a simulação, considera-se como canal um cabo par-trançado de alta qualidade (CAT5), típico de instalações DSL, com comprimento de $100$m~, cuja resposta em frequência (apresentada na Figura \ref{fig:ch6_canal_freqresponse}) é conhecida através de medição. Além disto, utiliza-se parâmetros característicos da tecnologia G.fast, destacando-se a largura de banda de aproximadamente $106$~MHz, espaçamento tonal ($\Delta f$) de $51.75$~kHz, $2048$ dimensões complexas ($\bar{N}=2048$ ou $N=4096$), máxima potência de transmissão de $4$~dBm e ruído de fundo com densidade espectral de potência de $-135$~dBm/Hz. 

\begin{figure}[htbp]
\centering
\includegraphics[width=0.95\textwidth]{./figs/simulacao_ch6_canal_freqresponse}
\caption{Espectro de magnitude do canal simulado.
\label{fig:ch6_canal_freqresponse}}
\end{figure}

O algoritmo de carregamento de bits utilizado adota um \textsl{gap} de SNR à capacidade de  $\Gamma = 10.75$, valor típico de instalações DSL atuais \cite{maes2012}, o qual inclui uma margem de ruído de $6$~dB para uma taxa de erro de bits (BER) alvo de $10^{-7}$. Este algoritmo consiste no algoritmo de Levin-Campello para maximização de taxa de bits \cite{campello1999}, o qual implementa a otimização de carregamento discreto de bits (diferentemente das soluções de enchimento com água, as quais resultam em carregamentos reais). Este algortimo baseia-se na análise da energia incremental que cada bit adicional demanda para ser transportado em cada subcanal \cite{ee379c}. Com esta informação, o algoritmo coloca iterativamente cada bit adicional no subcanal que requer menor energia incremental, até que o limite de energia total do símbolo seja satisfeito (ver problema de maximização de taxa de bits na Seção \ref{subsec:bit_loading}). Mais especificamente, o algoritmo implementa os processos de eficientização (\emph{efficientizing}) e estreitamento de energia (\emph{E-tightening}) descritos em \cite{ee379c,campelloisit1998}.

Inicialmente, considera-se a pré-codificação de ICI no transmissor e equalização de ISI no receptor. Para avaliar o desempenho do sistema, compara-se a taxa de bits de informação recebidos sem erro obtida no sistema com o esquema de mitigação e sem, isto é, para um sistema DMT que utiliza os prefixos-cíclicos regularmente. Os prefixos cíclicos utilizados na simulação são de 20, 40, 80, 160 e 320 amostras, os quais correspondem a $0.48\%$, $0.98\%$, $1.95\%$, $3.91\%$ e $7.81\%$ de $N$, respectivamente. Estes obedecem à relação entre taxa de bits e comprimento do prefixo cíclico introduzida na Figura \ref{fig:tradeoff_cp_bitrate}, o que implica na expectativa de maior taxa de bits para o prefixo cíclico de $0.48\%$. 

Para os prefixos cíclicos simulados, a PSD da distorção (soma de ISI e ICI) no símbolo recebido se mantém, na grande maioria dos subcanais, abaixo da PSD ruído de fundo (AWGN). Este comportamento está ilustrado na Figura \ref{fig:ch6_isi_ici_psd}, a qual apresenta a PSD da distorção PCI para cada um dos prefixos cíclicos\footnote{Devido ao elevado número de curvas, opta-se por não identifica-las, o que é irrelevante para o propósito de comparação com a PSD do ruído de fundo.} nos dois casos extremos de potência total de transmissão simulados ($0.2$~dBm e $4$~dBm), além da PSD do AWGN da simulação (em $-135$~dBm/Hz). Estas densidades espectrais de potência da distorção PCI são calculadas de acordo com a demonstração em \cite{perodling2002}. 

\begin{figure}[htbp]
\centering
\includegraphics[width=.95\textwidth]{./figs/simulacao_ch6_isi_ici_psd}
\caption{PSD da distorção PCI em comparação à PSD do AWGN.
\label{fig:ch6_isi_ici_psd}}
\end{figure}

Em relação à pré-codificação, é importante destacar que as linhas dos índices correspondentes aos subcanais das frequências DC e Nyquist ($k=0$ e $k=N/2$) nas matrizes de ISI e ICI no domínio da frequência ($\mathbf{T}_\text{ICI}$ e $\mathbf{T}_\text{ISI}$) são anuladas. Isto se deve ao fato de que estes subcanais não são utilizados para a transmissão, o que implica que suas distorções PCI são irrelevantes. Estas condições são aplicadas visando impedir o crescimento do fator de normalização ($\beta$) da matriz de pré-codificação normalizada (Equação \ref{eq:matriz_precoder_beta}), conforme será visto na Seção \ref{subsec:cond_iniciais}.

A Figura \ref{fig:simulacao_ch6_data_rate_135dbm_noise} apresenta o resultado da taxa de bits corretos recebidos para o sistema com pré-codificação e os sistemas com prefixos cíclicos (sem pré-codificação), além de um limite superior dado pela Equação \ref{eq:chan_capacity_gap} (na qual $\Gamma=10.75$). Nota-se que, como esperado, o desempenho obtido com a utilização de pré-codificação é significativamente superior.

\begin{figure}[htbp]
\centering
\includegraphics[width=1\textwidth]{./figs/simulacao_ch6_data_rate_135dbm_noise}
\caption{Taxa de bits alcançada com a pré-codificação.
\label{fig:simulacao_ch6_data_rate_135dbm_noise}}
\end{figure}

É importante destacar que a simulação cujo resultado está apresentado na Figura \ref{fig:simulacao_ch6_data_rate_135dbm_noise} implementa a equalização de ISI real, a qual, como mencionado na Seção~\ref{sec:isi_equalization}, sujeita-se à propagação de erros. Contudo, observa-se a partir do desempenho apresentado que a propagação de erro é, efetivamente, insignificante para o desempenho do sistema. Este argumento é ainda reforçado ao observar o resultado da simulação que força a remoção perfeita de ISI (sem erros), o qual é exatamente equivalente ao resultado da Figura \ref{fig:simulacao_ch6_data_rate_135dbm_noise}.

Finalmemente, ressalta-se que a pré-codificação de ICI e ISI no transmissor para o mesmo cenário de simulação também resulta exatamente no mesmo desempenho da Figura~\ref{fig:simulacao_ch6_data_rate_135dbm_noise}. É importante reforçar que, neste procedimento, as linhas da matriz $\mathbf{T}_\text{ISI}$ correspondentes aos indices das frequências DC e Nyquist ($k=0$ e $k=N/2$) são anuladas. Isto tipicamente garante que os símbolos pré-codificados (segundo a definição da Equação~\ref{eq:precoder_ici_isi}) respeitem os limites de potência, dado que a matriz mais influente na alteração de potência ($\mathbf{W'}$) já é normalizada.

