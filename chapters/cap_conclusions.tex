\chapter{Conclusion}
\section{Conclusion}

The development of this setup was mented to be general and be used in both
research and teaching environments. The process of implementing this setup went
through a myriad of fields in telecommunications and embedded systems, such as
communication protocols, embedded programming, electronics and many others,
being this setup a testbed for more complex processes inside the LTE band saves
a lot of time in development. \\

The FMComms2 board allows real-time and scalable change in parameters by
software or hardware signals, and it recalibrates and reconfigures itself if
needed so, which makes a very good transceiver board to be used in a C-RAN
environment.\\

This setup was extensively documented and can be used in digital communication
classes to show how a real radio frontend system is made, of course there is
much to improve, there is no dynamic synchronization between the FPGA and the
FMComms2, there is no communication protocol between the FPGA (BBP) and the
external world other than the FMComms2 such things are necessary to have a real
frontend but were outside the scope of this project which was just to evaluate a
setup for a scalable and dynamically configurable frontend.\\

Although the FMComms2 and the FPGA operates on different clocks, thus no real
synchronization was implemented, this setup has the minimal capability of
transmitting and receiving signals and reconfigure itself on real-time, the main
goal of this work was reached, however there is much to explore with these tools
and devices.

\section{Future Works}

Having finished this part of the work a natural sequel would be implementing
ethernet connection driver in the FPGA, making it possible to receive data from
ethernet and hand this to the transceiver board following the schematic on
figure X. This would be a challenge because there is a lot of things to
consider, but the most problematic of them all is synchronization
