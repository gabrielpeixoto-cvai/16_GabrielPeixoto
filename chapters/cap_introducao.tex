\chapter{Introduction}
\label{chap:intro}

The world is much more interconnected and the economy is growing wilder and
wilder, a lot of economical issues rise in different countries almost every
year. The climate changes are also a concern to the modern governments, thus the
idea of re-utilizing resources and  make these resources be more economically
and environmentally friendly are a goal for modern research and development.\\

Cloud computing is a paradigm which is growing every year in the companies and
developers, it is not a magic method to solve all the problems but it solves
some infrastructure problems of small and big companies, where companies does
not need to own computers or anything locally to operate, everything can be done
remotely and a more experienced company and staff can offer such infrastructure
or application as a service. Such idea of having everything as a service is very
attractive both economically  and environmentally because there is no waste of
resource, everything is scalable to the need of the client and upgradeable if
needed.\\

The Radio access technologies have been evolving from audio traffic to intensive
data traffic over the recent standards, because the mobile devices got a myriad
of functions which could only be executed by Personal Computers. However the
implementations of such networks demand a huge amount of resources, there is the
concern about how to develop and deploy these networks such as backwards
compatibility of the devices, because an operator would never deploy a network
which is not compatible with previous standards equipments, this is not
profitable. Having these ideas in mind the \textit{C-RAN} concept began to be
developed, such as cloud computing, now there is the technology for the RANs to
be scalable and configurable to the needs of the clients and operators, where
the baseband processing is all done on the cloud and the radio front-ends are
reconfigurable to handle different data and modulation outputs.\\

The reconfigurable fabric of the FPGA is very attractive to implement such radio
front-ends and other reconfigurable computing tasks, because it is flexible to
be reconfigured on-the-run and offers a really good processing and I/O
capabilities. Thus this work aims to evaluate the implementation feasibility of
a radio front-end LTE transmissions (LTE band) in a Cloud RAN environment
(adaptability and reconfigurability).\\

This work is implemented in a Xilinx Virtex 7 FPGA and using the transceiver
FMComms2 from analog devices, this transceiver can be reconfigured in real time
and runs up to 6 GHz band, it has been chosen because of such reconfigurability
properties. This text is composed by 6 chapters divided in 5 parts.\\

The second part is a literature review of some theoretical background used in
this work development, it is composed by two chapters, the second chapter
[\ref{chap:sdr}] describing Cloud and Software defined radio, because
reconfigurability and scalability is desired in SDR field. The third chapter
[\ref{chap:lte}] gives an overview in Digital Communication Systems and LTE,
describing the basic blocks of Digital Communications Systems and briefly
talking about LTE standard and how it is implemented.\\

The third part is the core of the work, the implementation chapter
[\ref{chap:implementation}]. In this chapter the functionalities of both FPGA
and Transceiver Board shall be explained, the development will be described in
terms of how it was implemented in both FPGA logic and software drivers.\\

The fourth part reports all the results of this work
[\ref{chap:results}]. The configuration results report how the transceiver
board communication and configuration were done. The simulation results aim
to show the VHDL blocks simulation prior to hardware implementation and at last
the transmission results aim in report how

The last part focuses in the conclusions and future works [\ref{chap:conclusion}],
which aim to report everything learned from this work and what can be done to
improve the transmission/reception quality.
