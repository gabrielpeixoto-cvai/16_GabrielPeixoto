\chapter{Introduction}
\label{chap:intro}

The world is much more interconnected and the economy is growing wilder and
wilder, a lot of economical crisis rise in different countries almost every
year. The climate changes are also a concern to the modern governments, thus the
idea of re-utilizing resources and  make these resources be more economically
and environmentally friendly are a goal for modern research and development.\\

Cloud computing is a paradigm which is growing every year in the companies and
developers, it is not a magic method to solve all the problems but it solves
some infrastructure problems of small and big companies, where companies does
not need to won computers or anything locally to operate, everything can be done
remotely and a more experienced company and staff can offer such infrastructure
or application as a service. Such idea of having everything as a service is very
attractive both economically  and environmentally because there is no waste of
resource, everything is scalable to the need of the client and upgradeable if
needed.\\

The Radio access technologies have been evolving from audio traffic to
intensive data traffic over the recent standards, because the mobile devices
got a myriad of functions which could only be executed by Personal computers,
however the implementations of such networks demand a huge amount of resources
and with the economy becoming harder every year, there is the concern about how
to develop and deploy such networks and of course backwards compatibility of
these networks, because an operator would never deploy a network which is not
compatible with previous standards equipments, this is not profitable. Having
these ideas in mind the Centralized RAN idea began to be developed, such as
cloud computing, now there is the technology for the RANs to be scalable and
configurable to the needs of the clients and operators, where the baseband
processing is all done on the cloud and the radio front-ends are reconfigurable
to handle different data and modulation outputs.\\

The FPGA is very attractive to implement such radio front-ends and other
reconfigurable computing tasks, because it is flexible too be reconfigured
on-the-run and offers a really good processing and I/O capabilities. Thus this
work aims to evaluate the implementation feasibility of a radio front-end to
LTE transmissions (LTE band) in a Cloud RAN environment (adaptability and
reconfigurability).\\

This work is implemented in a Xilinx Virtex 7 FPGA and using the transceiver
FMComms2 from analog devices, this transceiver can be reconfigured in real time
and runs up to 6 GHz band, it has been chosen because of such reconfigurability
properties. This text is composed by 6 chapters divided in 5 parts.\\

The first part aims to introduce the environment in which this work was planned
with this introduction chapter. The second part is a literature review of
everything theoretical used in this work development, it is composed by two
chapters, the second chapter describing Cloud and Software defined radio
\ref{chap:sdr}, because this reconfigurability property has been widely
explored in software defined radio field. The third chapter is about Digital
communications and LTE, describing the basic blocks of digital communications
systems and briefly talking about LTE standard and how it is implemented
\ref{chap:lte}.\\

The third part is the core of the work, the implementation, in this chapter
the functionalities of both FPGA and transceiver shall be explained and the
work development will be described and how it was implemented in both FPGA
logic and software drivers \ref{chap:implementation}.\\

The fourth part reports all the results obtained in such work.The configuration
results report how the transceiver board communication and configuration were
succeed, the simulation results aim to show the VHDL blocks simulation prior to
hardware implementation and at last the transmission results \ref{chap:results}.

The last part is the conclusions and future works, which aim to report
everything learned from this work and what can be done to improve the
transmission/reception quality or communication \ref{chap:conclusion}.
