\chapter{Introduction}
\label{chap:intro}

%The world is much more interconnected and the economy is growing wilder and
%wilder, a lot of economical issues rise in different countries almost every
%year. The climate changes are also a concern to the modern governments, thus the
%idea of re-utilizing resources and  make these resources be more economically
%and environmentally friendly are a goal for modern research and development.

The scalability and reconfigurability of telecommunication systems is very
interesting for the companies and operators in the current economic needs,
re-utilization is the key to decrease deployment and equipment update. This work
aims to evaluate the feasibility of a radio-fronthaul System for
\textit{Long-Term Evolution} (LTE) transmissions in a reconfigurable and scalable
environment.

Cloud computing is a paradigm which use is getting more common in companies and
developers, it solves some infrastructure problems of small and big companies.
Companies do not need to own computers or anything locally to operate,
everything can be done remotely and a more experienced company and staff can
offer such infrastructure or application as a service. Such idea of having
everything as a service is very attractive both economically  and
environmentally because there is no waste of resource, everything is scalable to
the need of the client and upgradeable if needed.

The Radio access technologies have been evolving from audio traffic to intensive
data traffic over the recent standards, because the mobile devices got a myriad
of functions which could only be executed by \textit{Personal Computers} (PC).
Such networks demand a huge amount of resources. Hence, there are concerns about
how to develop and deploy these networks such as backwards compatibility of the
devices, because an operator would never deploy a network which is not
compatible with previous standards equipments, this is not profitable. Having
these ideas in mind the C-RAN concept began to be developed,  where the
resources for the RANs are scalable and configurable to needs of the clients and
operators, and where the baseband processing is all done on the cloud and the
radio frontends are reconfigurable to handle different data and modulation
outputs.

The reconfigurable fabric of the \textit{Field Programmable Gate Array} (FPGA)
technology is very attractive to implement such radio front-ends and other
reconfigurable computing tasks, because it is flexible to be reconfigured
on-the-run, offers a really good parallel processing and I/O capabilities. Thus
this work aims to evaluate the implementation feasibility of a radio frontend
LTE transmissions in a Cloud RAN environment (adaptability and
reconfigurability).

This work is implemented in a Xilinx Virtex 7 FPGA and using the transceiver
board \textit{FMCOMMS2} from \textit{Analog Devices}, this transceiver can be
reconfigured in real time and runs up to 6 GHz band, it has been chosen because
of such reconfigurability properties.

The remainder of this text is organized in 4 parts as follows:

Part II is a literature review of some theoretical background used in this work
development, divided in three chapters. Chapter \ref{chap:sdr} offers an
overview of Cloud and Software defined radio, because reconfigurability and
scalability are features desired in SDR field and in this work. In Chapter
\ref{chap:lte} there is an overview in Digital Communication Systems and LTE,
since this work shall focus on LTE  frequency band transmissions.

Part III is the core of the work, Chapter \ref{chap:implementation} describes
the implementation of this work's setup and gives an overview of functionalities
of both FPGA and Fmcomms2 shall be explained. The development will be described
in terms of how it was implemented in both FPGA logic and software drivers.

Part IV reports all the results of this work in Chapter
\ref{chap:results}. The configuration results report how the transceiver board
communication and configuration were done. The simulation results aim to show
the VHDL blocks simulation prior to hardware implementation and at last the
transmission results aim in report the integrity of the transmitted signals.

Part V presents the conclusions and future works in Chapter
\ref{chap:conclusion}, which aim to report everything learned from this work and
what can be done to improve the transmission/reception quality.
